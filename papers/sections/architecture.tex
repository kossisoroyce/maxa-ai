\section{System Architecture}
\label{sec:architecture}

\subsection{Overview}
The Maxa architecture implements an \textit{eternal inference} framework that maintains persistent context across interactions while ensuring low-latency responses. As shown in Figure~\ref{fig:architecture}, the system integrates a local Qdrant vector database with OpenAI's GPT-4-turbo to achieve both performance and scalability. This hybrid approach enables efficient context retrieval and maintenance while leveraging state-of-the-art language understanding.

\subsection{Core Components}

\subsubsection{Qdrant Vector Database}
The system employs a local Qdrant instance deployed via Docker for efficient vector storage and retrieval. Key features include:
\begin{itemize}
    \item Real-time vector similarity search
    \item Efficient memory management for long-term context storage
    \item Optimized for low-latency retrieval in conversational contexts
\end{itemize}

\subsubsection{LLM Integration}
The architecture leverages OpenAI's GPT-4-turbo as its primary language model, chosen for its:
\begin{itemize}
    \item Advanced reasoning capabilities
    \item Strong few-shot learning performance
    \item Robust handling of long-context conversations
\end{itemize}

\subsubsection{Memory Management}
The memory system implements a hierarchical structure with:
\begin{itemize}
    \item Short-term cache for immediate context
    \item Working memory for active conversation tracking
    \item Long-term storage in Qdrant for persistent context
\end{itemize}

\begin{figure*}[t]
    \centering
    \begin{tikzpicture}[node distance=1.2cm, auto, scale=0.9, transform shape]
        % Nodes with smaller font
        \node[draw, rectangle, minimum width=2.5cm, minimum height=0.8cm, font=\small] (input) {Input Layer};
        \node[draw, rectangle, minimum width=2.5cm, minimum height=0.8cm, below=of input, font=\small] (memory) {Memory Module};
        \node[draw, rectangle, minimum width=2.5cm, minimum height=0.8cm, right=2.5cm of input, font=\small] (reasoning) {Reasoning Engine};
        \node[draw, rectangle, minimum width=2.5cm, minimum height=0.8cm, right=2.5cm of memory, font=\small] (temporal) {Temporal Module};
        \node[draw, rectangle, minimum width=2.5cm, minimum height=0.8cm, below=1.2cm of memory, font=\small] (profile) {User Profile};
        \node[draw, rectangle, minimum width=2.5cm, minimum height=0.8cm, right=2.5cm of profile, font=\small] (output) {Output Layer};

        % Arrows with arrowhead style
        \draw[->, >=stealth, thick] (input) -- (reasoning);
        \draw[->, >=stealth, thick] (input) -- (memory);
        \draw[->, >=stealth, thick] (memory) -- (temporal);
        \draw[->, >=stealth, thick] (memory) -- (profile);
        \draw[->, >=stealth, thick] (profile) -- (output);
        \draw[->, >=stealth, thick] (temporal) -- (reasoning);
        \draw[->, >=stealth, thick] (reasoning) -- (output);
    \end{tikzpicture}
    \caption{Overview of the Maxa architecture showing the interaction between core components. The system maintains persistent state through its memory and temporal modules while enabling natural interactions via the reasoning engine.}
    \label{fig:architecture}
\end{figure*}

\subsection{Core Components}

\subsubsection{Theory of Mind Module}
The Theory of Mind (ToM) module maintains a rich representation of the user's mental state, including their beliefs, desires, and intentions. This module is responsible for:
\begin{itemize}
    \item Tracking user preferences and interests over time
    \item Modeling the user's emotional state and its evolution
    \item Maintaining relationship dynamics and trust levels
    \item Generating personalized responses based on the user model
\end{itemize}

\subsubsection{Temporal Awareness}
The temporal awareness module enables the system to reason about time and maintain context across interactions. Key features include:
\begin{itemize}
    \item Event scheduling and reminders
    \item Temporal pattern recognition in user behavior
    \item Contextual awareness of past and future events
    \item Adaptive response timing based on user availability patterns
\end{itemize}

\subsubsection{Consciousness Layer}
The consciousness layer provides meta-cognitive capabilities, including:
\begin{itemize}
    \item Self-monitoring of system performance
    \item Reflection on past interactions
    \item Adaptive learning from feedback
    \item Management of system goals and subgoals
\end{itemize}

\subsection{Data Flow}
The system processes each user interaction through a pipeline that includes:
1. Input parsing and preprocessing
2. Sentiment and intent analysis
3. Context retrieval from memory
4. Response generation
5. User model update
6. Learning and adaptation

This modular architecture allows for flexible extension and adaptation to different domains and use cases.
