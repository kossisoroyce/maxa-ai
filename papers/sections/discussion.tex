\section{Discussion}
\label{sec:discussion}

Our evaluation of Maxa reveals several important insights into the challenges and opportunities of building persistent AI systems. In this section, we analyze the results, discuss the implications of our findings, and identify areas for future improvement.

Our evaluation of Maxa reveals several important insights into building cognitive architectures for personalized AI assistants. The results demonstrate that integrating Theory of Mind capabilities significantly enhances the quality of human-AI interactions.

\subsection{Key Findings}

\subsubsection{Personalization}
The ability to maintain and utilize a rich user model proved crucial for personalization. Users responded positively to the system's capacity to remember past interactions and adapt its behavior accordingly. However, we observed diminishing returns with excessive personalization, suggesting the need for careful balance.

\subsubsection{Temporal Awareness}
The temporal reasoning capabilities allowed Maxa to maintain context across interactions, leading to more coherent conversations. This was particularly evident in scenarios requiring scheduling and follow-up discussions.

\subsubsection{Scalability}
While the architecture performed well with our test user base, we identified potential scalability challenges in the current implementation, particularly in the vector similarity search component during peak loads.

\subsection{Comparison with Existing Work}
Compared to existing systems like BlenderBot~\cite{roller2021recipes} and Meena~\cite{adigwe2020meena}, Maxa offers more sophisticated personalization through its integrated cognitive modules. However, it requires more computational resources, which may limit deployment on resource-constrained devices.

\subsection{Ethical Considerations}
The system's ability to model user behavior raises important privacy concerns. We implemented several safeguards:
\begin{itemize}
    \item Data minimization: Only necessary data is collected
    \item User control: Users can view and delete their data
    \item Anonymization: All stored data is pseudonymized
\end{itemize}

Despite these measures, ongoing work is needed to address potential biases in the model's responses and ensure fair treatment of all users.
