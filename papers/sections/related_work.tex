\section{Related Work}
\label{sec:related}

Our work builds upon several key areas of research in artificial intelligence and cognitive science. This section reviews the most relevant literature in these domains.

\subsection{Cognitive Architectures}
Cognitive architectures provide a unified framework for modeling intelligent behavior. Prominent examples include ACT-R~\cite{anderson2004integrated}, Soar~\cite{laird2012soar}, and Sigma~\cite{rosenbloom2016sigma}. These architectures typically include modules for perception, memory, and action selection. Maxa builds upon these foundations while focusing specifically on the challenges of conversational AI and personalization.

\subsection{Theory of Mind in AI}
Theory of Mind (ToM) refers to the ability to attribute mental states to oneself and others. Recent work has explored ToM in AI through various approaches, including neural network-based methods~\cite{rabinowitz2018machine} and hybrid symbolic-neural architectures~\cite{shridhar2020alfred}. Maxa's ToM module extends these approaches by incorporating temporal dynamics and emotional intelligence.

\subsection{Personalized Conversational Agents}
Personalization in conversational AI has been approached through techniques such as memory networks~\cite{sukhbaatar2015end}, persona-based models~\cite{zhang2018personalizing}, and reinforcement learning~\cite{li2016deep}. Maxa distinguishes itself by integrating these approaches within a unified cognitive architecture that maintains a persistent user model across interactions.

\subsection{Temporal Reasoning in AI}
Temporal reasoning is crucial for maintaining context in extended conversations. Recent work has explored temporal knowledge graphs~\cite{jia2021temporal} and neural-symbolic approaches~\cite{sinha2020clutrr}. Maxa's temporal awareness module builds upon these ideas while optimizing for real-time interaction in conversational settings.
